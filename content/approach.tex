\subsection{Methodology}

\begin{itemize}
  \item formal proofs
  \item simulations https://github.com/hans-m-song/harbergers-tax 
  \begin{itemize}
    \item async? 
    \item how probabilistic?
    \item static intervals?
  \end{itemize}
  \item variability - whether or not to participate, amount to bid, changes in pools computational share, nodes having different compute power
  \item people withdrawing/depositing?, people joining/leaving?
  \item auction for each chunk owned or single chunk per round?
  \item achieving nash equilibrium
\end{itemize}

\subsection{Milestones}

\begin{itemize}
  \item resultant economy - profits for pool, profits for participants
  \begin{itemize}
    \item profitable - nice (quantitative results)
    \item not as profitable - measure increased security vs loss of profit (need some qualitative work too)
  \end{itemize}
\end{itemize}

\begin{itemize}
  \item formally model
  \item optimise to maximise profits
  \item implement simulation of the model
  \item visualisation of simulation results
  \item derive the nash equilibrium?
\end{itemize} 

\subsection{Limitations}

\begin{itemize}
  \item simulation
  \begin{itemize}
    \item modeling can only rely on randomization (although could fit to something like gaussian)
    \item 
  \end{itemize}
  \item what if other pools buy chunks in this network? prove this is more profitable or that more people are willing to join this network instead of contributing to other pool
  \item how to get pools to opt into this over doing a 51% attack? how to get individuals to opt in?
\end{itemize}

\subsection{Contributions}