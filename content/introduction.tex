\paragraph{} At it's core, blockchain is about creating a public ledger that ensures immutability through decentralization. While there have been many implementations, all platforms require overcoming the issue of reaching consensus in a distributed trustless environment. Bitcoin \cite{nakamoto2009} is the platform which spurred the demand for innovation in blockchains with Satoshi Nakamoto's innovative Proof of Work (PoW) consensus algorithm. Bitcoin now stands at the forefront of crytocurrencies with market capitalization fluctuating around 100 billion USD \cite{bitcoinmarketcap2020}, utilizing an estimated 53TWh annually \cite{cambridge2020} as of March 2020.

% \todo[inline]{remove above paragraph? move to background?}

\paragraph{} The security of the PoW consensus algorithm is established through cryptographic puzzles which are associated with each block. These are solved by \textit{miners}, people or companies that dedicate computational power to guess the solution for a reward. Dedicating more resources gives miners a higher probability of guessing the solution. As Blockchains utilizing PoW rose in popularity, the number of competing miners seeking to profit from the rewards rose alongside. While there are still individual miners, the bulk of computation stems from miners that have contributed their computational power to a \textit{pool}, a congregation of miners lead by a pool supervisor who coordinates work and distributes rewards. This results in similar payouts to individual miners but with reduced variance. This leads to large portions of computational power aggregated under the authority of singular entities, effectively centralizing computation.

\paragraph{Problem} With four pools contributing almost 60\% of the total hash-rate for Bitcoin \cite{bitcoinpools2020} at the time of this writing and a real possibility for pools to exceeded 50\% (such as GHash in 2014\todo{how do i cite this?}). The only limiting factor is the pools goodwill, potential public backlash and a shared incentive to maintain the high valuation of the currency. As each pool has a centralized authority orchestrating operations, the scenario in which they collude to execute a 51\% attack is a possibility. Additionally, pool supervisors are able to censor transactions by excluding them from the blocks they mine. 

\paragraph{Related Works} There has been several proposed solutions to resolve this vulnerability. The two phase PoW \cite{bastiaan2015} introduces an additional puzzle that requires the use of a private key to sign the work done. While this method would discourage outsourcing, it essentially negates any possibility of public pools that miners rely on for reduced variance. The P2Pool is an attempt at creating a completely decentralized mining pool (i.e. without a supervisor) but it is riddled with technical issues such as performance, scalability, and higher variance compared to other pools. Smartpool \cite{smartpool2017} is another proposed solution in which the pool is orchestrated via a smart contract on the Ethereum blockchain. This is a valid solution that may solve the centralization of authority for public pools, however there is no incentive for private pools to utilize such a method. NiceHash is a cryptocurrency hash power broker in which members can rent hashing power to contribute to a pool of their choice. They make no attempts to address the issue of authority centralization, rather, their platform encourages it.

\todo[inline]{GetBlockTemplate - miners dont actually care?}
\todo[inline]{collateralized smart contracts \cite{chepurnoy2020} - miners dont want more risk?}

\paragraph{Solution} This project introduces and verifies an economic model inspired by the Harberger Tax \cite{posnerweyl2017} (refer to section \ref{section:harbergerstax}) that has a focus on encouraging the separation the authority from pool supervisors to external participants, negating the vulnerability introduced by the centralization of authority. This is achieved by creating a sharing economy in which individual miners or pools can offer their computational power to a public marketplace as \textit{chunks} from which external participants or other miners may purchase to become \textit{owners}. Through this marketplace, miners will be consistently rewarded with taxation on their contributed power while owners can have a chance at the block reward. Additionally, this means that pools no longer need to adhere to the rule of having less than 50\% of the global hash-rate as the authority is in the hands of the owners. The aim of the model is to attract pools to utilize the model as contributors (\textit{miners}) rather than participants (\textit{owners}). 
\todo[inline]{bring up the nash equlibrium - equate model to a game \\
The model can essentially be considered a non-cooperative game}
\todo[inline]{expand use cases \\
While this proposed solution is modelled with PoW in mind, it has the potential to be applied to other consensus algorithms}

% \paragraph{} proposed solution

% \begin{itemize}
%   \item decentralization of authority
%   \item separate miners from block reward / an alternative to corporations (Antpool) - mining empty blocks / selfish mining
%   \item pool manager can focus on efficiency / deduping as they are incentivized to attract more participants
% \end{itemize}

% \paragraph{} methodology

% \begin{itemize}
%   \item problem: centralization of computation (or authority)
%   \item proposed solution: separate computation from authority
%   \item incentive: new economy is more profitable/attractive to participate in
%   \item method 1: computation all owned by single entity - sell computation, tax transactions
%   \item method 2: computation is pooled by many - less rewards for more computation owned
% \end{itemize}