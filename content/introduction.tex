% A clear definition of the topic, goals and relevance will leave the reader with no doubt about the intended coverage and contribution of the thesis. 
% The definitions should include a project outline and clear statement of purpose. 

\subsection{Background Coverage}

% literature review

\subsubsection{Blockchains}

% general background of what blockchains are
\begin{itemize}
  \item blocks, achieving consensus - hashes, byzantine fault tolerance
  \item scoping to pow or broader?
  \item centralisation of computation - 51\% attack
  \item incentives to mine - block reward - (hardware cost + power cost)
  \item solo vs pooled mining
\end{itemize}

\subsubsection{Harbergers Tax}

\begin{itemize}
  \item incentivisation of proper distribution
  \item tax on transaction instead of interval?
\end{itemize}

\subsection{Justification} % statement of purpose?

\begin{itemize}
  \item currently, almost 60\% of hashrate is from 4 pools %cite this
  \item decentralisation of computation - may be less profitable but more secure
  \item separate miners from blockreward / an alternative to corporations (Antpool) - mining empty blocks / selfish mining
  \item pool manager can focus on efficiency / deduping calculations as they are incentivised to attract more participants
\end{itemize}

\subsubsection{Definitions}

\paragraph{Blockchain}

\begin{itemize}
  \item block reward, block time, market cap
\end{itemize}

\paragraph{Pool}

\begin{itemize}
  \item available chunks, compute power vs world, trade intervals, tax
\end{itemize}

\paragraph{Participant}

\begin{itemize}
  \item funds, owned chunks, price
\end{itemize}