\subsection{Coverage}

\subsubsection{Blockchains}

% general background of what blockchains are
\begin{itemize}
  \item blocks, achieving consensus - hashes, byzantine fault tolerance
  \item scoping to pow or broader?
  \item centralization of computation - 51\% attack
  \item incentives to mine - block reward - (hardware cost + power cost)
  \item solo vs pooled mining
\end{itemize}

\subsubsection{Harbergers Tax} \label{section:harbergerstax}

\begin{itemize}
  \item incentivization of proper distribution
  \item tax on transaction instead of interval?
\end{itemize}

\paragraph{} Harberger's Tax is an economic policy that aims to prevent unequal distribution of property through two rules:

\begin{itemize}
  \item Owners assign a self-assessed value to their property and pay a proportional tax.
  \item The owner is unable to prevent anybody from purchasing their property at their previously set price.
\end{itemize}

\subsubsection{Nash Equilibrium}

\begin{itemize}
  \item defining parameters
  \item defining strategies
  \item deriving the equilibria
\end{itemize}

\subsubsection{Definitions}

\paragraph{Blockchain}

\begin{itemize}
  \item block reward, block time, market cap
\end{itemize}

% PoW relies on a non-trivial amount of adequately distributed computational power to ensure the immutability of it's public ledger. While this 
% payout is dependent on pool policy

\paragraph{Pool}

\begin{itemize}
  \item available chunks, compute power vs world, trade intervals, tax
\end{itemize}

\paragraph{Participant}

\begin{itemize}
  \item funds, owned chunks, price
\end{itemize}

\subsection{Literature Review} % prior work, what they've achieved, what their limitations are

\begin{itemize}
  \item smartpool (only on ethereum relying on smart contracts - cant be used for bitcoin)
  \item p2pool (seems dead?)
\end{itemize}