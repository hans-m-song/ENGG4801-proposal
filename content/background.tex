\subsection{Blockchains}

\begin{itemize}
  \item blocks, achieving consensus - hashes, byzantine fault tolerance
  \item scoping to pow or broader?
  \item centralization of computation - 51\% attack
  \item incentives to mine - block reward - (hardware cost + power cost)
  \item solo vs pooled mining
\end{itemize}  

\paragraph{} A Blockchain is an append-only set of records that are linked cryptographically to 

\paragraph{} Consensus Algorithms are at the core of any blockchain and are an attempt at overcoming the issue of overall system consistency of the blockchain despite faulty or malicious participants in a distributed, trustless environment. The primary obstacle of consensus algorithms is the Byzantine Generals Problem: a scenario in which allied generals geographically separated by the enemy are attempting to communicate without the message being caught and tampered with. PoW overcomes this issue by creating a cryptographic puzzle that depends on the previous block in the chain and the new transactions that are to be in the next block.

\paragraph{} Attempting to rewrite history or faking blocks requires having enough computational power to solve the puzzles of each related block and overtake the creation of the next block, creating a \textit{hard fork}. However, there is no punishment for collusion, nor is behaving honestly always the most profitable method of participation. 

\subsection{Harbergers Tax} \label{section:harbergerstax}

\begin{itemize}
  \item incentivization of proper distribution
  \item tax on transaction instead of interval?
\end{itemize}

\paragraph{} Harberger's Tax is an economic policy that aims to prevent unequal distribution of property through two rules:

\begin{itemize}
  \item Owners assign a self-assessed value to their property and pay a proportional tax.
  \item The owner is unable to prevent anybody from purchasing their property at their previously set price.
\end{itemize}

\paragraph{} When the policy is enacted, participants are able to set a price on their property through self-assessed valuation

\subsection{Nash Equilibrium}

\begin{itemize}
  \item defining parameters
  \item defining strategies
  \item deriving the equilibria
\end{itemize}

\subsection{Formal Definitions}

\paragraph{Blockchain}

\begin{itemize}
  \item block reward, block time, market cap
\end{itemize}

% PoW relies on a non-trivial amount of adequately distributed computational power to ensure the immutability of it's public ledger. While this 
% payout is dependent on pool policy

\paragraph{Pool}

\begin{itemize}
  \item available chunks, compute power vs world, trade intervals, tax
\end{itemize}

\paragraph{Participant}

\begin{itemize}
  \item funds, owned chunks, price
\end{itemize}