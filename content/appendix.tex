\appendix

\begin{landscape}
  \section{Project Plan Timeline} \label{appendix:gantt}

  % gantt chart variables
  \newcommand{\ganttstart}{2020-04-02}
  \newcommand{\ganttend}{2020-11-09}
  \newcommand{\ganttdays}{222}
  \newcommand{\ganttthesisend}{2020-06-20}
  \newcommand{\ganttseminar}{2020-05-22}
  \newcommand{\ganttdemonstration}{2020-10-23}
  \newcommand{\ganttformalmodelstart}{\ganttstart}
  \newcommand{\ganttformalmodelend}{2020-04-30}
  \newcommand{\ganttsimulationstart}{\ganttstart}
  \newcommand{\ganttsimulationend}{2020-06-15}
  \newcommand{\ganttsimulationlogicstart}{\ganttstart}
  \newcommand{\ganttsimulationlogicend}{2020-06-01}
  \newcommand{\ganttsimulationvisualstart}{2020-05-15}
  \newcommand{\ganttsimulationvisualend}{\ganttsimulationend}
  \newcommand{\ganttoptimisestart}{2020-06-20}
  \newcommand{\ganttoptimiseend}{2020-08-15}
  \newcommand{\ganttnashstart}{2020-08-20}
  \newcommand{\ganttnashend}{2020-09-22}

  \begin{ganttchart}[
      milestone label font=\tiny,
      group label font=\tiny,
      title label font=\tiny,
      bar label node/.style={text width=3cm,align=right,font=\scriptsize\RaggedLeft,anchor=east},
      milestone label node/.style={text width=2cm,align=right,font=\scriptsize\RaggedLeft,anchor=east},
      group label node/.style={text width=3cm,align=right,font=\scriptsize\RaggedLeft,anchor=east},
      x unit = 0.0968cm,
      time slot format=isodate
    ]{\ganttstart}{\ganttend}
    \hbadness=10000
    \gantttitlecalendar{month=shortname} \\

    \ganttbar{Project Proposal}{\ganttstart}{\ganttstart} \\
    \ganttbar{Formal Modelling}{\ganttstart}{\ganttformalmodelend} \\
    \ganttmilestone{Seminar}{\ganttseminar} \\
    \ganttgroup{Implement Model Simulation}{\ganttsimulationstart}{\ganttsimulationend} \\
    \ganttbar{Implement Model Logic}{\ganttsimulationlogicstart}{\ganttsimulationlogicend} \\
    \ganttbar{Visualize Model Results}{\ganttsimulationvisualstart}{\ganttsimulationvisualend} \\
    \ganttbar{Optimise Model Parameters}{\ganttoptimisestart}{\ganttoptimiseend} \\
    \ganttbar{Derive Nash Equlibrium}{\ganttnashstart}{\ganttnashend} \\
    \ganttmilestone{Demonstration}{\ganttdemonstration} \\
    \ganttbar{Thesis}{\ganttthesisend}{\ganttend}    

    \ganttlink{elem6}{elem7}
    \ganttlink{elem3}{elem6}
    \ganttlink{elem3}{elem9}
  \end{ganttchart}
\end{landscape}

\section{Simulation Code} \label{appendix:code}

\todo[inline]{is this necessary?}

\subsection{Entities}

\paragraph{Pool}
\paragraph{Blockchain}
\paragraph{Participant}
\paragraph{Orchestrator}
\paragraph{Bidder}

\subsection{Actions}

\paragraph{auction}
\paragraph{taxCollection}

\subsection{Data collection}

\paragraph{Metrics}
\paragraph{Analysis and Postprocessing}
\paragraph{Visualization}

\newpage\todo[inline]{notes: remove this at the end}

A miner will use several items in an attempt to solve the puzzle:

\begin{itemize}
  \item $T$ - A collection of unverified transactions.
  \item $t$ - the digest of a Merkle tree created from $T$
  \item $h_p$ - The hash of the previous block.
  \item $n$ - a random string otherwise known as the \textit{nonce}.
  \item $\lambda$ - the difficulty of the puzzle (where $\lambda > 1$) where $\lambda$ is the required number of zeroes padding the solution.
  \item $H$ - the hash function chosen by the miner that outputs 256 bits based on the input.
\end{itemize}
A potential hash is found via the function $h = H(t \| h_p \| n)$, when an appropriate nonce is chosen, $h$ should satisfy the condition $h \leq \frac{2^{246}}{\lambda}$ to be a suitable solution.
